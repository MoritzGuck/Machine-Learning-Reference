\documentclass[main.tex]{subfiles}

\begin{document}
\section{Kernel methods}
Kernel methods help in using linear decision boundaries on non-linearly separable data by morphing the feature space. They can also help in disentangling data for clustering. Kernels can incorporate domain knowledge into your model.

\subsection{Introduction to Kernels}
Kernels can be seen in two ways: As similarity measures between data-points or transformations of the data-points into a higher dimensional space. 
For two points x and y a Kernel is given by:
\begin{equation}
    K(x,y) = \sum_{i=1}^n h_i(x) h_i(y) = \langle {h(x),h(y)} \rangle,
\end{equation}
where $h(x)$ is a transformation-function and $\langle \cdot \rangle$ is the inner product. 
% todo complete this section

\subsection{Kernel SVM}



\end{document}